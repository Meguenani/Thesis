\begin{abstract}
{\setstretch{1.0}
\vspace{3mm}
The intended goal of this thesis is to bring new insights for developing robotic systems capable of safely sharing their workspace with human-operators. Within this context, the presented work focuses on the control problem. The following questions are tackled:
\begin{itemize}
\item for reactive control laws, i.e., control problems where the task to be performed is not known in advance but discovered \textit{on-line}, how is it possible to guarantee for every time-step the existence of a solution to the control problem? This solution should allow the robot to accomplish at best its prescribed task and at the same time to strictly comply with existing constraints, among which, constraints related to the physical limitations of its actuators and joints.
\item How to integrate the human-operator in the control loop of the robot so that physical contact can safely  be engaged and de-engaged?
\end{itemize}
Regarding the first point, our work arises as the continuity of previous results developed by S\'ebastien Rubrecht during his PhD thesis. S\'ebastien Rubrecht introduced the concept of \textit{constraints incompatibility} for robots reactively controlled at the kinematic-level. The problem of \textit{constraints incompatibility} appears for example when the formulation of the constraint on an articular position of a robot does not account for the amount of deceleration producible by its actuator. In such case, if the articular position constraint is activated tardively, the system may not have sufficient time to cope with the imposed joint position limit considering its bounded dynamic capabilities. \\
In our work, the problem of \textit{constraints incompatibility} is tackled for robots controlled at the dynamic-level (torque control). New mathematical formulations for the \textit{articular position} and \textit{joint velocity} constraints that account for both the deceleration and jerk capabilities of the actuators of the robot are introduced. The \textit{optimal} time needed for the system to cope with the different limits of these constraints is computed. When included in a QP formulation of the control problem, the introduced new expressions trigger the activation of the constraints at the right time considering the dynamic capabilities of the robot. The existence of a constraint compliant and optimal solution as joint position, velocity, torque/deceleration and also jerk limits are all respected at the same time and at every time-step is guaranteed. Consequently the \textit{viability} of the state of the robot is also ensured.

Regarding the second point, for a given shape of the robot, two main parameters are usually considered for safety: the impact force generated at collision and the contact force generated after the establishment of physical contact. The most generic way to consider and express these parameters is to use an energetic formulation of the safety problem. Indeed, energy is a universal quantity that can describe all the safety related physical phenomena occurring during human-robot interaction. An impact force for example is directly proportional to the amount of kinetic energy dissipated at collision. The contact force on the other hand derives from the amount of potential energy that accumulates in the controller of the robot during physical contact. In our work, safety during human-robot interaction is handled by modulating the amount of energy instantaneously displayed by the robot. The main contributions are as follows:
\begin{itemize}
\item the formulation of kinetic and potential energy based safety indicators that represent the degree of danger of the robot at collision and during physical contact.
\item The formulation of energy based safety criteria that bound the dynamics of the robot during different interaction phases: at the approach of the human-operator, at collision, during physical contact and finally when contact is released.
\item The expression of the introduced indicators and criteria as inequality constraints related to the torque control input of a robotic manipulator.
\item The introduction of the concept of ``task energy profile'' to track the amount of energy used to accomplish a considered repetitive task in the most optimal way. This profile is then used to formulate a constraint on the instantaneous amount of energy the robot is allowed to exhibit at every time-step. When implemented in the control scheme, the ``task energy profile'' related constraint makes the robot comply to any deliberate or non-deliberate contact with its environment. 
\end{itemize}
For the validation of the proposed algorithms, beyond the theoretical contributions, simulation and experimental applications are performed on a KUKA LWR4 robotic manipulator.}
%Large and sudden changes in the torques of the actuators of a robot are highly undesirable and should be avoided during robot control as they may result in unpredictable behaviours. Multi-objective control system for complex robots usually have to handle multiple prioritized tasks while satisfying constraints. Changes in tasks and/or constraints are inevitable for robots when adapting to the unstructured and dynamic environment, and they may lead to large sudden changes in torques. Within this work, the problem of task priority transitions and changing constraints is primarily considered to reduce large sudden changes in torques. This is achieved through two main contributions as follows.
%
%Firstly, based on quadratic programming (QP), a new controller called Generalized Hierarchical Control (GHC) is developed to deal with task priority transitions among arbitrary prioritized task. This projector can be used to achieve continuous task priority transitions, as well as insert or remove tasks among a set of tasks to be performed in an elegant way. The control input (\textit{e.g.} joint torques) is computed by solving one quadratic programming problem, where generalized projectors are adopted to maintain a task hierarchy while satisfying equality and inequality constraints.
%
%Secondly, a predictive control primitive based on Model Predictive Control (MPC) is developed to handle presence of discontinuities in the constraints that the robot must satisfy, such as the breaking of contacts with the environment or the avoidance of an obstacle. The controller takes the advantages of predictive formulations to anticipate the evolutions of the constraints by means of control scenarios and/or sensor information, and thus generate new continuous constraints to replace the original discontinuous constraints in the QP reactive controller. As a result, the rate of change in joint torques is minimized compared with the original discontinuous constraints. This predictive control primitive does not directly modify the desired task objectives, but the constraints to ensure that the worst case of changes of torques is well-managed.
%
%The effectiveness of the proposed control framework is validated by a set of experiments in simulation on the Kuka LWR robot and the iCub humanoid robot. The results show that the proposed approach significantly decrease the rate of change in joint torques when task priorities switch or discontinuous constraints occur. 
\vspace{7mm}

\textbf{Keywords:} \textit{Redundant Robots, Dynamic Control, Safe Human-Robot Interaction, Kinetic energy, Potential Energy, Energy-Based Constraints.}
\end{abstract}

%robots redondants, contrôle dynamique, Commande prédictive, Optimisation quadratique, commande en couple, continuités des contraintes


