\chapter{Introduction}
\label{chap:intro}
\setfigurepath{figures/Intro}
%=========================================================================
%The intended goal of this thesis is to bring new insights for developing robotic systems capable of safely sharing their workspace with human operators. Within this context, the presented work focuses on the control problem. The following questions are tackled:
%1) for reactive control problems, i.e. control problems where the task to be performed is not known in advance but discovered on-line, how is it possible to guarantee at every time-step the existence of a solution to the control problem? This solution should allow the robot to accomplish at best its prescribed task and at the same time to strictly comply with existing constraints, among which, constraints related to the physical limitations of its actuators and joints.
%2) How to integrate the human operator within the control loop of the robot so that physical contact can be safely engaged and de-engaged?
%Regarding the first point, our work arises as the continuity of previous results developed by Sébastien Rubrecht during his PhD thesis.
%Sébastien Rubrecht introduced the concept of constraints incompatibility that appears for example when the formulation of an articular position related constraint does not take into account the amount of deceleration producible by the joint actuator. In this case, if this constraint is activated tardively, the system may not be able to cope with its position limit considering its bounded reaction capability. In his thesis, Rubrecht solved this problem for a robot reactively controlled at the kinematic level. In our work, the problem of constraints incompatibility is extended to robots controlled at the dynamic level. New mathematical formulations for the joint position and velocity constraints are introduced considering both the deceleration and jerk capabilities of the robot actuators. When included in a QP formulation of the control problem, these new expressions guarantee the existence of a constraint compliant and optimal solution at all time as joint position, velocity, torque/deceleration and also jerk limits are all respected at the same time and at every time-step.
%6:06
%Regarding the second point: for a given shape of the robot, two main parameters are usually considered for safety: the impact force generated at collision and the contact force generated after the establishment of physical contact. The most generic way to consider and express these parameters is to use an energetic formulation. Indeed, energy is a universal entity that can describe all physical phenomena occurring during human-robot interaction. An impact force for example is directly proportional to the amount of kinetic energy dissipated at collision. The contact force on the other hand derives from the amount of potential energy generated within the human-robot system during physical contact. In our work, safety during human-robot interaction is guaranteed by modulating the amount of energy instantaneously deployed by the robot. The main contributions are as following:
%1) the formulation of kinetic and potential energy based safety indicators that represent the degree of danger of the robot at collision and during physical contact.
%2) The formulation of safety criteria that bounds the dynamic of the robot during different interaction phases: at the approach of the human operator, at collision, during physical contact and finally when this contact is released.
%3) The expression of these indicators and criteria as inequality constraints related to the torque control input of a robotic arm.
%4) The introduction of the concept of "task energy profile" to track the amount of energy used to accomplish a considered repetitive task in the most optimal way. This profile is used to constrain the instantaneous amount of energy the robot is allowed to exhibit at every time-step.
%The resulting controller renders the robot compliant to any deliberate or non-deliberate contact with its environment.
%For the validation of the proposed algorithms, beyond the theoretical contributions, simulation and experimental applications are performed on a Kuka LWR-4 serial robot. 
\section{Motivation}
The present thesis is partially supported by the RTE company through the RTE/UPMC Chair of ``\textit{Intervention Robotics}" held by Vincent Padois. The intended goal of this chair is to bring new insights for the development of intervention robotic systems and assistant robots capable of safely interacting with human operators. Within this context, the work presented hereby focuses on the control problem. M\'ecaniques \\
Domains of application for robots are evolving from purely structured industrial contexts to the human world as intervention machines and assistants to aid humans in the completion of manual tasks. This has a direct impact on the formulation of the control problem that must be completely reconsidered. Indeed, when industrial robots are aimed for tasks that are relatively simple (e.g. \textit{pick and place manipulation}), repetitive, within perfectly known static and protected environments; Service and collaborative robots are confronted to more challenging scenarios: unknown, constrained and dynamic environments with possible deliberate and non deliberate interactions with humans. 
\section{Research questions and contributions}
The goal of the presented work is to seek answers for the following questions: 
\begin{itemize}
\item For reactive control problems, i.e. control problems where the task to be performed is not known in advance but discovered on-line, how is it possible to guarantee every time-step the existence of a solution to the control problem ? This solution should allow the robot to accomplish at best its prescribed task and at the same time to strictly comply with existing constraints, among which, constraints related to the physical limitations of its actuators and joints.
\item How to integrate the human operator within the control loop of the robot so that physical contact can be safely engaged and de-engaged ? 
\end{itemize} 
Regarding the first point, our work arises as the continuity of previous results developed by Sébastien Rubrecht during his PhD thesis.
Sébastien Rubrecht introduced the concept of constraints incompatibility that appears for example when the formulation of an articular position related constraint does not take into account the amount of deceleration producible by the joint actuator. In this case, if this constraint is activated tardively, the system may not be able to cope with its position limit considering its bounded reaction capability. In his thesis, Rubrecht solved this problem for a robot reactively controlled at the kinematic level. In our work, the problem of constraints incompatibility is extended to robots controlled at the dynamic level. New mathematical formulations for the joint position and velocity constraints are introduced considering both the deceleration and jerk capabilities of the robot actuators. When included in a QP formulation of the control problem, these new expressions guarantee the existence of a constraint compliant and optimal solution as joint position, velocity, torque/deceleration and also jerk limits are simultaneously respected at the same time and at every time-step. \\

Regarding the second point, for a given shape of the robot, two main parameters are usually considered for safety: the impact force generated at collision and the contact force generated after the establishment of physical contact. The most generic way to consider and express these parameters is to use an energetic formulation. Indeed, energy is universal  and can describe most physical phenomena occurring during human-robot interaction. An impact force for example is directly proportional to the amount of kinetic energy dissipated at collision. The contact force on the other hand derives from the amount of potential energy generated within the human-robot system during physical contact. In our work, safety during human-robot interaction is guaranteed by modulating the amount of energy instantaneously deployed by the robot. The main contributions regarding safe human-robot interaction are as following:
\begin{itemize}
\item The formulation of kinetic and potential energy based safety indicators that represent the degree of danger of the robot at collision and during physical contact.
\item The formulation of safety criteria that bounds the dynamic of the robot during different interaction phases: at the approach of the human operator, at collision, during physical contact and finally as the contact is released.
\item The expression of these indicators and criteria as inequality constraints related to the torque control input of a robotic arm.
\item The introduction of the concept of "task energy profile" to track the amount of energy used to accomplish a considered repetitive task in the most optimal way. This profile is used to constrain the instantaneous amount of energy the robot is allowed to exhibit at every time-step.
The resulting controller renders the robot compliant to any deliberate or non-deliberate contact with its environment.
\end{itemize}
For the validation of the proposed algorithms, beyond the theoretical contributions, simulation and experimental applications are performed on a KUKA LWR4 serial robot. 
%This work arises as a continuity of previous results developed by S\'ebastien Rubrecht \cite{rubrecht2011contributions} and Joseph Salini \cite{salini2012}. Within the context of the Telemach industrial project, S\'ebastien Rubrecht introduced the concept of constraints incompatibility for a robotic arm reactively controlled, 
%
%
%
% i.e. the reformulation of different static constraints (e.g. articular positions) to ensure at every time step the existence of an optimal solution for the control problem. On the other hand, Joseph Salini formalized a general framework to describe the control problem for humanoid robots using LQPs (Linear Quadratic Program). This same approach will be used throughout the presented work. 
%
%These different contributions provide an initial response to the fixed goals; However, the presence of dynamic constraints (e.g. mobile obstacles) and human operators around the robot highlight other concerns that must be dealt with. The highest priority in the control of any type of human-friendly robot is to prevent of the actions of the robot from causing any harm to the human operator. A safe human-robot collaboration requires switching between different control modes (before/after physical contact), the formulation of safety indicators to reflect the amount of danger towards the human operator, the formulation of safety criteria representing the bounds on the dynamic behaviour of the robot and finally expressing all of these as constraints related to the control parameters. Bottom line, for a given shape, two parameters are to be considered for safety: the impact force created at the collision instant between the robot and the human operator and the contact forces generated after the establishment of physical contact. The most generic way to include and express these parameters is to use an energetic formulation. Indeed, energy is a universal entity that can describe almost all the physical phenomena occurring during human-robot interaction. For example, the impact force is directly related to the amount of kinetic energy dissipated at the collision and the contact forces derive from the amount of potential energy generated during physical contact. 
%
%Two types of constraints are considered during interaction; Static constraints where the maximum allowed limit is constant and dynamic ones where the limiting bound variates depending on other parameters (e.g. human-robot distance). For both kinds, the reaction capabilities of the robots actuators must be considered to guaranty every time step a feasible solution of the control problem. This comes down to activating the constraints on the control parameters at the right instant to ensure sufficient time to cope with the constraints. 
%
%\section{Contributions}  
%The main contributions of the presented work are as following :
%\begin{itemize} 
%\item The formulation of an energy based safety indicator that allows the modulation  of the impact force and the contact forces.
%\item The formulation of safety criteria that bound the dynamic behaviour of the robotic system considering different situations: The approach of the human operator, the establishment of physical contact and finally the releasing of contact.
%\item A theoretical study on the reformulation of two articular static constraints (position and velocity limitations) while taking into account the reaction capabilities of the robots actuators (available braking torque and jerk).
%\end{itemize} 
%For the validation of the developed algorithms, beyond the theoretical contribution, simulation and experimental application are performed on a KUKA LWR4 serial robot. For the distance acquisition, a point cloud based algorithm is used with multiple Microsoft Kinect devices. 
\section{Structure of the manuscript}
The presented manuscript is organized in three main chapters: Kinetic energy based safety indicator for human robot interaction, a constraints compatibility study for static constraints with application on a KUKA LWR4 and finally potential and kinetic energy based safety constraints formulation. 
\section{Related Publications}
Some of the novel developments introduced in this work have been peer-reviewed and validated with the publications listed hereafter:
\begin{itemize}
\item Meguenani, A., Padois, V., \& Bidaud, P. (2015, September). Control of robots sharing their workspace with humans: an energetic approach to safety. In Intelligent Robots and Systems (IROS), 2015 IEEE/RSJ International Conference on (pp. 4678-4684). IEEE.
\item Meguenani, A., Padois, V., Da Silva, J., Hoarau, A., \& Bidaud, P. (2017). Energy based control for safe human-robot physical interaction. In 2016 International Symposium on Experimental Robotics (Vol. 1). Springer.
\end{itemize}


